\chapter[Hypotheses and goals]{Hypotheses and goals}
\label{ch3}

\section{Hypotheses}
\label{ch3_1}

Taking the previously mentioned facts into account, it is hypothesized that quantitative acoustic analysis of voice/speech signals can be used to robustly and complexly describe and identify HD in PD, and to indirectly assess other non-speech symptoms of PD. Specifically, it is assumed that parametrization of voice/speech deficits in HD and application of statistical analysis and/or modern machine learning techniques is capable of estimating the values of clinical rating scales that are conventionally used to assess motor and non-motor symptoms of PD at the baseline, as well as in the horizon of two years.

\section{Goals and objectives}
\label{ch3_2}

The main goal of this doctoral thesis is to \textit{investigate possibilities of using quantitative objective evaluation of HD, employing modern clinically interpretable speech parametrization, statistical analysis and machine learning techniques, in direction of PD identification and assessment}. More specifically, this thesis has five main objectives that can be briefly summarized as follows:
\begin{enumerate}
	\item \textit{Robust computerized quantification of HD manifestations in PD}\,--\,to use modern clinically interpretable speech parameterization techniques to quantify manifestations of HD in the area of phonation, articulation, prosody and speech fluency that are known to occur with idiopathic PD.
	\item \textit{Complex analysis and identification of dysprosody in HD}\,--\,to study dysprosody in HD and to investigate an influence of prosodic demands such as precise control of speech melody variability during recitation or modulation of stress in speech, on computerized identification of HD.
	\item \textit{Assessment of non-speech symptoms of PD at the baseline}\,--\,to analyse the possibilities of using acoustic analysis of HD to estimate the scores of a~variety of clinical rating scales that are nowadays being commonly used to assess motor and non-motor symptoms of PD at the baseline.
	\item \textit{Assessment of gait freezing in PD in the horizon of two years}\,--\,to analyse the possibilities of using acoustic analysis of HD at the baseline for predicting the change in the severity of gait freezing in PD in the horizon of two years. 
	\item \textit{Analyse pathological mechanism shared by HD and gait freezing in PD}\,--\,to investigate if there are any pathological mechanisms shared by voice/speech disorders in HD and freezing of gait in PD.
\end{enumerate}

