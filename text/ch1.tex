\chapter[Parkinson's disease]{Parkinson's disease}
\label{ch1}

This chapter deals with the state of knowledge in the field of PD analysis. PD is one of the most frequent and complex neurological disorders that does affect people at an epidemic rate worldwide. This chapter describes its history, epidemiology, pathophysiology, manifestations, risk factors, diagnosis, assessment, monitoring, treatment, therapies, etc. It specifically points out to limitations of the current clinical approach to diagnosis and assessment of the disease. It subsequently provides a~description of the novel non-invasive a~para-clinical computerized approaches taking advantage of modern signal processing techniques, state-of-the-art statistical and machine learning algorithms, etc. that can be used to provide clinicians with additional supportive information for the early and accurate diagnosis, remote and objective assessment, prediction, and monitoring of this fatal disease.

\section{State of knowledge}
\label{ch1_1}

\subsection{Brief history of the disease}
\label{ch1_1.1}

The oldest description of so called \textit{parkinsonian symptoms}\footnote{Clinical symptoms such as tremor at rest, bradykinesia, muscular rigidity, postural instability, etc. caused by some form of brain dysfunction that accompany a~family of disorders summarized under the term Parkinsonism.} dates back to $5\,000$ B.C. in the ancient India~\cite{Manyam1990}. There are other references to these symptoms~\cite{Diaz2004} as well, e.g. ancient Chinese sources, ancient Egyptian sources, bible, and many more \cite{Zhang2006, Lees2007, Goetz2011}. Even though such mentions can be found all over the world throughout the history, the foundation of knowledge about PD as we know it today was first laid in $1817$ by the English physician James Parkinson in his milestone work named ``An Essay on the Shaking Palsy''~\cite{Parkinson2002}. In this work, James Parkinson analysed and systematically described medical conditions of $6$~individuals and based on his observations, he established the term \textit{paralysis agitans} (shaking palsy) to describe the symptoms that are nowadays being recognized and well-documented to accompany idiopathic PD.

Almost 60 years later, in $1877$, the globally recognized term \textit{Parkinson's disease} was established by the French neurologist and professor of anatomical pathology, also known as ``the founder of modern neurology'', Jean-Martin Charcot. Charcot and his students were the first to make a~distinction between muscular rigidity, weakness and bradykinesia in PD, and provided a~comprehensive clinical description of the arthritic changes, dysautonomia, and pain occurring with this disease~\cite{Charcot1877}. Few years later, in $1888$, the British neurologist William Gowers reported an influential work~\cite{Gowers1886} comprehensively describing his personal experience with $80$ patients suffering from PD. Further description of morphology and pathophysiological changes related to PD was reported by Richer and Meige in $1895$~\cite{Richer1895}. The anatomy and damage caused to \textit{substancia nigra} in the midbrain and the pathology of PD was described in more details by Konstantin Tretiakoff~\cite{Tretiakoff1919} and Charles Foix~\cite{Foix1925} in $1919$ and $1925$, respectively. In $1953$, the complete pathophysiological analysis of the brain-stem lesions in PD was performed by Greenfield and Bosanquet~\cite{Greenfield1953}. Next the biochemical mechanisms behind the pathophysiology of PD were further described in $1957$ by the Swedish Nobel Prize laureate Arvid Carlsson~\cite{Carlsson1957}, and finally one of the most famous works dealing with PD was published in $1967$ by Hoehn and Yahr~\cite{Hoehn1967} that introduced the stages of this disease in the course of its progression. The same year, levodopa (L-dopa) was first used as a~medication for PD~\cite{Fahn2008}. Until today, there has been a~large number of other key studies addressing the epidemiology, pathology, diagnosis, assessment, treatment of PD, etc. such as~\cite{Hornykiewicz1970, Rijk1997, Hornykiewicz1998, Jankovic2001} and many others. However, more comprehensive study of history and the evolution of knowledge about this disease is left to the reader.

\subsection{Description of the disease}
\label{ch1_1.2}

Even thought it has been approximately $200$ years since James Parkinson provided his description of \textit{paralysis agitans}, the exact aetiology of PD (the underlying cause of its onset) is still not fully understood, and its conceptualisation continues to evolve. At present, neuropathology of PD is being described as follows: PD is a~chronic idiopathic disorder characterized by the specific pattern of progressive loss or degeneration of dopaminergic neurons in basal ganglia, especially in the \textit{substancia nigra pars compacta}~\cite{Hornykiewicz1998} (SNpc), but also in the other regions of the brain~\cite{Dickson2012} with the development of $\alpha$-synuclein-containing Lewy bodies within the surviving neurons~\cite{Forno1996}. There is a~variety of other neurodegenerative disorders that share similar parkinsonian-like manifestations such as Lewy Body Dementia (LBD)~\cite{Mckeith2007}, Multiple System Atrophy (MSA)~\cite{Wenning2004}, Progressive Supranuclear Palsy (PSP)~\cite{Cole2003}, etc. A~sub-set of these disorders, specifically: Lewy body dementia, PD, PD dementia) are nowadays being summarized under the name Lewy Bodies Disorders\footnote{Term that is used to signify that there is an underlying $\alpha$-synuclein deposits in the brain, which results into autonomic, cognitive, behavioral or motor dysfunction, etc.} (LBDs). However, in the context of this thesis, only idiopathic PD is considered.

Up to this day, the gradual dopaminergic\footnote{Dopamine is an organic chemical that functions as a~neurotransmitter responsible for transmitting nerve impulses within the brain that allow for motor control and movement coordination.} deficiency within the basal ganglia has been recognized as a~major cause of a~very heterogeneous set of parkinsonian symptoms~\cite{Brodal2003}. It leads to a~malfunction of the central nervous system (CNS) that is no longer capable of coordinating muscle movements properly, which consequently results into a~large variety of associated motor symptoms. The primary motor symptoms of PD that are manifested predominantly on upper and lower extremities~\cite{Hornykiewicz1998} comprise tremor at rest (unintentional rhythmic and oscillatory movements such as trembling/shaking of a~part of the body with frequency bellow $6$\,Hz; present in approximately 70\,\% of PD patients \cite{Hoehn1967, Hughes1993, Louis2001}), progressive bradykinesia\footnote{Bradykinesia is used synonymously with: hypokinesia (a~poverty of movement. e.g. parkinsonian micrographia), and akinesia (an~absence of movement, e.g. poor facial expressions, etc.).} (slowness of initiation of voluntary movement with progressive reduction in speed and amplitude of repetitive actions; present in most patients with PD \cite{Yanagisawa1989, Berardelli2001}), muscular rigidity (resistance of the muscles to passive manipulation; present in approximately $89$\,\%--$99$\,\% of PD patients \cite{Hoehn1967, Hughes1993, Louis1997}), and postural instability (difficulties in adapting the posture, poor balance, and unsteadiness; present in most patients with PD~\cite{Horak2005}, however it is also present in a~variety of other disorders and therefore it has low diagnostic specificity). Next, motor symptoms of PD such as dysarthria, dysphagia, gait freezing~\cite{Hornykiewicz1998, Ho1999a}, etc. can be present. In addition to the motor symptoms, patients with PD also develop a~variety of non-motor symptoms such as the neuropsychiatric symptoms (depression, cognitive dysfunction, dementia, etc.), sensory symptoms (olfactory deficits, visual dysfunction, etc.), gastrointestinal symptoms (constipation, excess salivation, dysphagia, etc.), autonomic symptoms (bladder disturbances, changes in sweating, orthostatic hypotension, etc.), as well as sleep disturbances, etc.~\cite{Hoehn1967} that are nowadays being described as a~result of $\alpha$-synuclein deposits in the brain and in the periphery. As reported by the previous studies, in approximately 70\,\% of PD patients these symptoms have asymmetric onset~\cite{Hughes1993}. In summary, considering the large number of symptoms accompanying PD, it is evident how significantly this disease reduces the independence, social integration, mental and physical condition, and quality of life of patients suffering from it, and how dramatically it increases the requirements for their caregivers as well.

According to the previously published studies~\cite{Rijk2000}, PD is the second most frequent neurodegenerative disorder. The prevalence/incidence rate\footnote{Prevalence rate is the fraction of newly diagnosed patients at any given time. Incidence rate is the number of new patients per population at risk in a~given time period.} of PD has been estimated to approximately $1.5\,\%$/$1.24$~\cite{Savica2016} respectively (prevalence rate being approximately $1.5$ times larger in population of men as opposed to the population of women \cite{Baldereschi2000, Haaxma2007}) for people aged over $65$ years~\cite{Rijk1997} (PD is rare in the young population however there are still cases of PD appearing in much younger patients~\cite{Colcher1999}). Moreover, as reported by Schrag et al.~\cite{Schrag2002}, another $20\,\%$ of people with PD are currently misdiagnosed/undiagnosed. Regarding the risk factors, in fact, literature suggests that ageing is the most critical risk factor of PD onset~\cite{Elbaz2002}. In addition to the patient's advancing age, other factors such as the family history, exposure to pesticides/chemicals, drug abusing, environmental stress, traumatic brain injury, etc. have also been reported to promote the neuropathology associated with PD. According to \cite{Hernan2002, Ascherio2016}, protective factors include regular tobacco smoking, coffee/tea drinking, consumptions of antioxidants, physical activity, and others. Unfortunately, there is no scientifically validated preventive course reducing the risk of PD onset.

At present, there is no definite cure for PD. Nevertheless, various medication and/or therapeutic strategies aiming primarily at the treatment of its motor manifestations have been developed. Levodopa, which is based on the compensation of the dopaminergic loss in the nigrostriatal system, has been the most widely used form of PD medication for a~long time, and it still remains to be a~standard way of medication for alleviating the typical parkinsonian symptoms. Even though the cardinal motor symptoms can be relieved reasonably well, complications of long-term dopaminergic drug use are also known to develop \cite{Grosset2009, Gardian2010}. There are other medications besides L-dopa such as dopamine agonists, monoamine oxidase B and catechol-O-methyl transferase inhibitors, etc., as well as treatment method for PD such as duodopa pump, Deep Brain Stimulation (DBS), and ones under investigation such as repetitive Transcranial Magnetic Stimulation (rTMS) \cite{Herzog2003, Rodriguez2005, Benabid2009, Deuschl2006}. As can be seen, there are many methods in the clinician's toolbox for the treatment of the disease. Unfortunately, at the end of the day, none of these methods can cure PD, and clinicians are eventually limited to alleviating its symptoms and to maintaining patients' quality of live for as long as possible.

\section{Conventional approaches and limitations}
\label{ch2_2}

\subsection{Diagnosis}
\label{ch1_2.1}

PD, as well as other neurodegenerative diseases, does not start suddenly. It is a~progressive and continuous process that appears gradually with an increasing severity over time. In the early stages of the disease, initial motor and non-motor symptoms of the associated neurodegeneration have already been present, however they have not yet advanced to the stage in which they can be conclusively and definitively diagnosed using the classic clinical methodology. According to the International Parkinson and Movement Disorder Society (MDS), early PD can be divided into the following $3$ developmental stages:
\begin{enumerate}
	\item Preclinical PD\,--\,in this stage, the neurodegenerative processes have already began, but there are no evident symptoms that can be clinically diagnosed (however some imaging or biomarker abnormalities are present \cite{Wu2011, Chahine2014}).
	\item Prodromal PD\,--\,in this stage, some of the parkinsonian symptoms are present, but they are still insufficient for a~define diagnosis (further imaging or biomarker abnormalities are present/amplified \cite{Stern2012, Berg2015}).
	\item Clinical PD\,--\,in this stage, the classical motor and non-motor parkinsonian symptoms have advanced to the stage in which they become explicit and finally sufficient for a~probable clinical diagnosis~\cite{Postuma2015}.
\end{enumerate}

As mentioned above, before the classical symptoms of PD can be clinically diagnosed, the neurodegenerative process has already commenced. In fact, at this stage, the process of dopaminergic degeneration has reached a~critical point in which as much as $60$\,--\,$70$\,\% of dopaminergic neurons had already been damaged \cite{RodriguezOrzo2009, Bernheimer1973}. According to the literature, the motor parkinsonism is the core feature of PD diagnosis. However, motor symptoms alone are not sufficient for the diagnosis as non-motor manifestations are present in most patients as well. As reported by the previous studies, non-motor symptoms can dominate the clinical presentation of PD, and in some cases, these symptoms can appear prior to the onset of the aforementioned cardinal motor ones~\cite{Postuma2012}. Moreover, early in the process of PD onset, it is very difficult to determine whether the observed symptoms and signs do actually indicate the presence of PD or they are caused by the presence of another disease with similar parkinsonian-like manifestations such as medication-induced parkinsonism, essential tremor, PSP, MSA, dementia, etc.

Even today, no objective diagnostic test which allows for definitive and conclusive diagnosis of PD has been developed. The gold-standard for PD diagnosis has been the presence of SNpc degeneration and Lewy pathology at post-mortem pathological examination. The conventionally used criteria according to the UK Parkinson's Disease Society Brain Bank for PD diagnosis are composed of the following steps~\cite{Hughes1992} (the full criteria can be seen in Appendix~\ref{tab:PD_clinical_diagnosis_criteria}; there are also other diagnostic criteria such as \cite{Gelb1999, Albanese2003, Postuma2015}, etc.): 
\begin{enumerate}
	\item Diagnosis of typical parkinsonian symptoms\,--\,presence of progressive bradykinesia in combination with at least one the following features: muscular tone (rigidity), $4$--$6$\,Hz resting tremor, postural instability (not caused by primary visual, vestibular, cerebellar, or proprioceptive dysfunction).
	\item Exclusion criteria for PD\,--\,one or more of the following features: history of repeated strokes with stepwise progression of parkinsonian features, history of repeated head injury, history of definite encephalitis, negative response to large doses of levodopa, etc.
	\item Supportive prospective positive criteria for PD\,--\,three or more of the following features are required for definite diagnosis of PD: unilateral onset, presence of resting tremor, progressive nature of disorder, great response to levodopa ($70$--$100$\,\%), etc.
\end{enumerate}

\subsection{Assessment and monitoring}
\label{ch1_2.2}

At present, there are no objectively measured characteristics and methods (i.\,e. biomarkers) for evaluating the disease progression and for quantifying the efficacy of treatment in PD~\cite{Antoniades2008}. The actual evaluation and monitoring of PD symptoms progression as well as the effect of the anti-parkinsonian treatment is achieved by the subjective assessment of the ability of patients to perform a~range of empirical tests during regular physical visits at the clinic~\cite{factor2007}. For this purpose, a~variety of standardized clinical rating scales evaluating motor and non-motor symptoms of PD has been developed. Amongst the most commonly used ones: Unified Parkinson's Disease Rating Scale\footnote{It has been criticized that the currently used UPDRS is confusing for capturing non-motor symptoms of PD and therefore the Movement Disorder Society has sponsored a~revision of it~\cite{Goetz2007}} (UPDRS; parts: I) Mentation, Behavior and Mood; II) Activities of Daily Living; III) Motor Examination; and IV) Complications of Therapy)~\cite{Fahn1987}, Non-Motor Symptoms Scale (NMSS)~\cite{Chaudhuri2007}, Beck Depression Inventory (BDI)~\cite{Beck2000, Beck1961}, Freezing Of Gait questionnaire (FOG-Q)~\cite{Giladi2000}, The REM sleep Behaviour Disorder Screening Questionnaire (RBDSQ)~\cite{Stiasny2007}, Mini-Mental State Examination (MMSE)~\cite{Folstein1975} or Addenbrooke's Cognitive Examination-Revised (ACE-R)~\cite{Larner2007}, etc. has been used.

Nevertheless, subjective assessment of PD severity often varies between clinicians due to inter-rater variability~\cite{Ramaker2002, Post2005}. A~para-clinical methods that would be able to estimate scores of previously mentioned scales and provide neurologists and clinical psychologists with a~quick and preliminary insight into motor and non-motor features of the examined patient are still missing. As reported by the previous studies~\cite{Tsanas2010, Tsanas2010b}, these methods can provide non-invasiveness, inexpensiveness, and most importantly, natural objectivity of the examination, minimizing the need for regular subjective clinical expertise.

\section{Novel a~para-clinical approaches}
\label{ch2_3}

In summary, currently established clinical diagnosis and assessment of PD is a~complex process, which however can not be made with 100\,\% certainty \cite{Hughes2002} (75\,\%--95\,\% of PD patients have their diagnosis confirmed on autopsy), and that heavily relies on the presence of both motor and non-motor symptoms, short-term positive response to dopaminergic medication, presence of levodopa-induced dyskinesia, etc. The actual clinical examination is performed in the medical environment under the supervision of skilled clinicians using a~battery of clinical tests (e.g. clinical rating scales~\cite{factor2007} assessing motor and non-motor deficits, etc.), imaging techniques (Magnetic Resonance Imaging (MRI), Single Photon Emission Computed Tomography (SPECT), Positron Emission Tomography (PET), etc.) \cite{Stern1989, Jankovic2008, Peran2010}, biomarkers (motor performance tests, oculomotor measurements, olfaction tests, biochemical measurements such as blood tests, evaluation of rapid eye movement (REM) sleep behavior disorder (RBD)), and inclusion/exclusion criteria~\cite{Hughes1992, Gelb1999, Albanese2003, Postuma2015}. The problem with this approach is that it is logistically demanding for patients as well as their caregivers, cost-ineffective, time-consuming, and so on. Considering the current trend of population ageing, we can imagine that more people gets older, the more serious this becomes. It gets even trickier when we also take into account the fact that each patient experiences the symptoms and reacts to the anti-parkinsonian medication/treatment individually~\cite{Grosset2009}. Therefore, to make the diagnosis as precise as possible, long periods of observation are often necessary.

Today, there is a~large number of smart devices such as smart phones, smart watches, and others, which can be used to record a~variety of biological signals such as voice/speech, gait, tremor of extremities, etc. These devices are therefore capable of providing clinicians with large amount of data about the health condition of the patient on a~daily basis, and without necessity of supervision or patient's presence at the clinic. So, if such data are merged with the data acquired using classical clinical examination, new possibilities of computer-based supportive diagnosis and assessment of PD using so called decision support system can be developed. Based on advanced signal processing techniques, robust mathematical modelling and statistical processing, such a~systems could in theory identify unique patterns of distinct parameters combinations derived from multimodal clinical and a~para-clinical biomarkers that would not reach relevant diagnostic accuracy when evaluated one by one separately. However, to reach this point, further research is necessary. In the frame of this thesis, usage of computerized acoustic analysis of voice/speech signals for quantification of hypokinetic dysarthria as a~frequent and disabling symptom of PD, and indirect assessment of other non-speech symptom of PD is considered.