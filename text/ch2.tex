\chapter[Hypokinetic dysarthria]{Hypokinetic dysarthria}
\label{ch2}
This chapter deals with the state of knowledge in the field of HD analysis. HD is a~frequent speech disorder associated with idiopathic PD with detrimental impact on the verbal communication and daily social life of patients suffering from it. The chapter describes its manifestations, diagnosis, assessment, monitoring, etc. It specifically points out to limitations of the current clinical approach to diagnosis and assessment of HD. It subsequently provides a~description of the novel approaches taking advantage of modern signal processing techniques, state-of-the-art machine learning algorithms, etc. that can be used to provide clinicians with additional supportive information for the early and accurate diagnosis, remote assessment, prediction and monitoring of this disease.

\section{State of knowledge}
\label{ch2_1}

Dysarthria is a~medical term used to collectively describe a~family of neuromuscular speech disorders associated with disturbance of phonorespiration (discoordinated respiration and insufficient airflow support), impaired control over laryngeal muscle function (presence of vocal tremor and irregular vocal folds' vibration pattern), increased vocal nasality, imprecise articulation of consonants, unnatural prosody, speech rate abnormalities, poor speech quality, etc. At present, there are six major types of dysarthria \cite{Mcneil1997, Murdoch1998, Kent2000, Rusz2014} that are related to an underlying neurologic condition, as well as the presence of deviant speech dimensions identified by clinical researchers Darley, Aronson, and Brown back in $1969$~\cite{Darley1969b}:
\begin{itemize}
	\item flaccid dysarthria (impaired lower motor neuron system)
	\item spastic dysarthria (impaired upper motor neuron system)
	\item ataxic dysarthria (damaged cerebellar system)
	\item hyperkinetic dysarthria (disorder of the extrapyramidal motor system)
	\item hypokinetic dysarthria (disorder of the extrapyramidal motor system)
	\item mixed dysarthria (a~combination of $2$/more dysarthria types)
\end{itemize}

Dysarthria is nowadays well-known to result from a~variety of conditions such as stroke, brain tumours/injuries, and most frequently, it has been shown to occur as a~symptom of a~large number of progressive neurological disorders\footnote{Resulting from central and/or peripheral nervous system abnormalities~\cite{Pinto2010}} such as Amyotrophic Lateral Sclerosis (ALS), Multiple Sclerosis (MS), Parkinson's disease (PD) \cite{Darley1969, Darley1969b, Darley1975, Murdoch1998}, etc. In the context of this thesis, only hypokinetic dysarthria in idiopathic PD~\cite{Brabenec2017} is considered.

According to the previously published studies~\cite{Ho1999a}, up to $90\,\%$ of patients suffering from idiopathic PD do eventually develop the distinctive motor speech disorder first described in $1969$ \cite{Darley1969, Darley1969b} that is nowadays known as hypokinetic dysarthria (HD). At present, HD is considered to be a~result of the defective motor execution of articulatory programs~\cite{Duffy2015} that is particularly attributed to subcortical neuropathology (synucleinopathy) disrupting basal ganglia-thalamocortical motor networks for speech production~\cite{Fujii2014}. From the perspective of its symptoms, HD is known to be manifested in all dimensions of human voice/speech production, specifically in: respiration, phonation, articulation, prosody, speech fluency, and faciokinesis \cite{Darley1969, Darley1975, Mekyska2011b_eng, Brabenec2017}. It is characterized by rigidity, bradykinesia, and reduced muscular control of the larynx, articulatory organs, and other physiological support mechanisms of human speech production~\cite{Kegl1999}. The following voice/speech disorders, which significantly contribute to reduced speech intelligibility, ability to communicate, and quality of life of patients with HD in general \cite{Pell2006, Hall2011}, have been observed: increased acoustic noise~\cite{Hornykiewicz1998}, reduced intensity of voice~\cite{Baker1998}, harsh and breathy voice quality \cite{Tsanas2010, Tsanas2010b}, increased voice nasality~\cite{Spencer2005}, reduced speech prosody i.\,e. monopitch, monoloudness, and speech rate disturbances~\cite{Brin2009, Skodda2009, Galaz2016}, reduced variability of the articulatory organs' mobility i.\,e. imprecise articulation of consonants~\cite{Roy2009, Gomez2018}, involuntary introduction of pauses~\cite{Moretti2003}, palilalia, i.\,e. rapid repetitions of words and syllables~\cite{Moretti2003}, sudden deceleration or acceleration in speech (bradyphemia/tachyphemia)~\cite{Gentil1995}, etc.

Even though HD has been known for almost fifty years, and it still remains to be one of the most common speech disorder under investigation, its underlying pathophysiological mechanisms are not yet fully understood. Accordingly, as the research in the field of PD and HD in particular has evolved over time, an understanding of the neurological changes responsible for communication impairment associated with HD has evolved as well. In the early days, the researchers thought that HD in PD could be attributed to the classical parkinsonian symptoms~\cite{Darley1969}. Only the time showed that the lack of effective pharmacological (dopaminergic) treatment and surgical intervention for alleviating the symptoms of HD do actually point to different mechanisms of speech impairment in PD \cite{Skodda2011c, skodda2014, Elfmarkova2016}. Moreover, it has been shown that symptoms of HD and speech prosody impairment in particular progress over time without any clear correlation to disease duration or global motor symptom scores \cite{Skodda2009, Skodda2011b}. Therefore, it seems that the non-dopaminergic pathways are being involved in the pathophysiology of HD, and some parallels can be drawn particularly between HD and gait problems \cite{Moreau2007, Cantiniaux2010}. Many other parallels can be drawn to link HD with non-dopaminergic mechanisms in PD but the important conclusion is that HD is a~complex disorder that have to be viewed from multiple perspectives.

Literature demonstrates the influence of PD in speech from early to advanced stages \cite{Rusz2011, Skodda2013}, but HD is mainly observed in mid to advanced stages of PD~\cite{Duffy2013}. But not only that, according to other sources~\cite{Tetrud1991}, there have been cases in which the family members of PD patients did observe changes in production and quality of voice/speech even before the disease has been conclusively diagnosed. Hence, it seems that when analysed accordingly, HD could be potentially used as an early and powerful marker for diagnosis and a~rich source of information for the assessment and monitoring of PD. 

\section{Conventional approaches and limitations}
\label{ch2_2}

Generally, voice/speech analysis can be approached by: a) perceptive clinical evaluation (i.\,e. the analysis is performed by a~skilled speech therapist according to a~variety of standardized protocols and clinical rating scales); or b) quantitative para-clinical evaluation (i.\,e. the computerized acoustic analysis of voice/speech signals is performed). Regarding the perceptive analysis of HD in PD, most neuroscience-oriented research in PD has been based on the Unified Parkinson’s Rating Scale (UPDRS), part III: Motor Examination, item $18$ for evaluation of speech production rated on a~$0$--$4$ scale~\cite{Fahn1987, Brabenec2017}. However, this is just a~screening and an insufficiently detailed measure of HD. Another clinical rating scales that are frequently being used to evaluate HD are Frenchay Dysarthria Assessment 2nd Edition (FDA-2)~\cite{Cardoso2017}, Robertson Dysarthria Profile (RDP)~\cite{Defazio2016}, etc. An alternative option is for instance a~visual analog scaling of speech impairment severity, e.\,g. for the Grandfather Passage, assessment of dysphonia using GRBAS (grade, roughness, breathiness, asthenia and strain) rating scale, and so on. There are more detailed scales available, such as the 3F test (for Czech speakers) as well. This instrument evaluates three major domains of dysarthria, including faciokinesis, phonetics, and photorespiration, with a~calculated dysarthric index ranging from $0$ (the most severe symptoms) to $90$ (no symptoms of dysarthria) points~\cite{Kostalova2013}. However, such diagnostic tools are usually available only in the country of their origin and have not been translated and validated for broader use across Europe or elsewhere.

Assessment of speech intelligibility~\cite{Yorkston1984} or that of the voice/speech handicap~\cite{Midi2008} is clinically meaningful, however, not very sensitive, particularly in cases of mild HD symptoms that may occur early in the course of the disease or even in prodromal stages of PD. Another drawback of the clinical approach to voice/speech quality assessment is its natural subjectivity. It might happen that if other examiner rated the patient, the results would be slightly different due to medical/mental condition, environmental factors, stress, etc. Finally, even a~trained therapist is limited to human sound perception (only sounds in the audible part of its spectrum can be perceived and rated). So, to provide therapists with additional supportive information about voice/speech disorders occurring with PD, and therefore make its assessment or diagnosis more accurate and objective, quantitative para-clinical evaluation of voice/speech signals has been investigated.

\section{Novel para-clinical approaches}
\label{ch2_3}

In quantitative para-clinical evaluation, also known as the objective acoustic analysis of voice/speech signals, an audio recording is usually digitized and then processed on a~computer. This process comprises speech parameterization, statistical analysis, mathematical modelling, etc. During the speech parameterization, signal properties that are important for speech pathology description are quantified. In addition to the pathology under focus, the parameterization is dependent on a~properly selected speech task. For a~quantification of various aspects of HD in PD, a~wide range of speech tasks have been employed (most commonly used ones):
\begin{itemize}
	\item sustained phonation of the vowels (/a/, /e/, /i/, /o/, and /u/)
	\item syllable repetition (diadochokinetic) tasks
	\item sentence repetition tasks
	\item reading tasks
	\item running speech (monologue)
\end{itemize}

To measure the quality of voice, sustained phonation of vowels has been the most frequently used speech task across the literature \cite{Tsanas2010b, Astrom2011, Hariharan2014, Smekal2015c, Naranjo2016}. More specifically, sustained phonation of the vowel /a/ is a~standard measure used to assess quality of phonation. During this particular speech task, a~speaker is asked to sustain phonation of a~vowel, attempting to maintain steady frequency and amplitude at a~comfortable level~\cite{Titze1994}. The advantage of this speech task in comparison with other commonly used speech tasks is its independence of articulatory and other linguistic confounds~\cite{Titze1994}. Moreover, it is also present in most of the available databases~\cite{Harar2017, Harar2018}. Next, to measure the quality of speech articulation, syllable repetitions have been used with great success \cite{Lowit2008, Skodda2011d, Skodda2013, Orozco2014a}. And finally, to measure the quality of speech prosody, various speech tasks such as sentence repetition tasks, reading tasks and/or running speech (monologue) have been used \cite{Skodda2011b, Rusz2013b, Bandini2015, Galaz2016}. To quantify the voice/speech disorders in HD, various parametrization techniques has been developed. In the frame of this thesis, these methods are referred to as \textit{acoustic features}. In general, the acoustic features can be roughly divided into $2$~main categories~\cite{Brabenec2017}: a) conventional ones; and b) non-conventional ones.

The conventional acoustic features are the most commonly used ones to describe voice/speech deterioration present in HD. These features provide a~community of researchers and clinicians with a~unique possibility of linking the values of these features with the specific manifestations of HD (these features are conceptually simple and therefore clinically interpretable~\cite{Smekal2015b}). Thus far, the conventional acoustic features have been used to describe: a) the impairment of phonatory aspects of speech using several variants of jitter and shimmer, intensity variations described by the standard deviation (SD) of the squared energy operator (SEO), or Teager–Kaiser energy operator (TEO), the SD of the time that vocal folds are apart and in collision \cite{Little2007, Gelzinis2008, Silva2009, Little2009, Tsanas2010, Tsanas2010b, Smekal2015c, Naranjo2016}, etc.; b) speech quality deterioration using harmonic-to-noise ratio (HNR), noise-to-harmonic ratio (NHR), glottal-to-noise excitation ratio (GNE), fraction of locally unvoiced frames (FLUF) \cite{Michaelis1997, Little2007, Gelzinis2008, Little2009, Smekal2015c, Naranjo2016}, etc.; c) impairment of speech prosody using the SD of fundamental frequency (F0SD), the relative SD of F0 (relF0SD), the variation range of F0 (F0VR), the relative VR of F0 (relF0VR), the SD of SEO (SEOSD), the SD of TEO (TEOSD), the variation range of SEO (SEOVR), the variation range of TEO (TEOVR), the relative VR of SEO (relSEOVR), the relative VR of TEO (relTEOVR) \cite{Galaz2016, Skodda2009, Skodda2010}, etc.; d) speech rate disturbances using total speech time (TST), net speech time (NST), total pause time (TPT), total speech rate (TSR), net speech rate (NSR), articulation rate (AR), percent pause time (PPT), speech index of rhythmicity (SPIR) \cite{Rusz2011, Galaz2016, Skodda2009, Skodda2010, Skodda2011b}, etc.; and e) impaired consonant articulation and tongue movement using frequencies and bandwidths of the first three formants (Fx and Bx), the formant centralization ratio (FCR), the vowel space area (VSA) and its logarithmic version (lnVSA), the vowel articulation index (VAI), the ratio of second formants of vowels [i] and [u] (Fi/Fu), DDK rate, DDK regularity, voice onset time (VOT) \cite{Forrest1989, Sapir2010, Rusz2011, Skodda2011, Rusz2013}, etc.

However, in more advanced stages of PD, the voice becomes aperiodic, noisy, irregular, and chaotic. Sometimes, this results in the inability of conventional acoustic features to capture useful clinical information about the underlying voice pathology. To deal with this problem, researchers have developed more complex and robust non-conventional acoustic features. Compared to the conventional ones, these features provide more precise HD identification and tracking. However, non-conventional acoustic features are in general less clinically interpretable \cite{Smekal2015b}. Amongst the most commonly used ones, features based on empirical mode decomposition (EMD) \cite{Smekal2015a, Smekal2015c, Tsanas2010, Tsanas2010b}, correlation dimension (CD), fractal dimension (FD) \cite{Shao2010, Vaziri2010, Orozco2013a, Orozco2013b}, Hurst exponent (HE), largest Lyapunov exponent (LLE) \cite{Vaziri2010, Orozco2013a}, approximate entropy (AE), sample entropy (SE), correlation entropy (CE), recurrence probability density entropy (RPDE) \cite{Henriquez2009, Little2007, Little2009, Orozco2013a}, mel-frequency cepstral coefficients (MFCC) \cite{Tsanas2010, Tsanas2010b, Smekal2015b, Naranjo2016}, detrended fluctuation analysis (DFA) \cite{Little2007, Little2009, Tsanas2010, Tsanas2010b}, pitch period entropy (PPE)~\cite{Little2009}, cepstral peak prominence (CPP)~\cite{Hillenbrand1995}, normalized noise energy (NNE)~\cite{Kasuya1986}, etc. have been analyzed.

Acoustic analysis of dysarthric speech was shown to be a~promising biomarker of PD~\cite{Tsanas2012, Mekyska2015} with a~great potential to objectively assess severity of PD~\cite{Skodda2009, Tsanas2010b, Smekal2015c}. As reported by the recent studies, acoustic analysis of speech in HD can provide clinicians with non-invasive and reliable methodology of PD examination that can be used in daily clinical practice for identification, assessment and monitoring of the progress of PD~\cite{Brabenec2017, Tsanas2010b} and also the efficiency of the treatment~\cite{Harel2004b, Rusz2013b}.