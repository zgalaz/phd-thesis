\chapter{Appendix}
\section{Parkinson's Disease diagnosis criteria}

\begin{table*}[htb!]
	\caption{UK Parkinson's Disease Society Brain Bank diagnosis criteria~\cite{Hughes1992}.}
	\footnotesize
    \centering
	\label{tab:PD_clinical_diagnosis_criteria}

	\begin{tabular}{l l}
		\hline\hline\noalign{\smallskip}
		Step & Diagnostic criteria \\
		\noalign{\smallskip}\hline\noalign{\smallskip}
		\textbf{1} & \textbf{Diagnosis of parkinsonian syndromme} \\
		  & bradykinesia \\
		  & one/more of the following: muscular rigidity, $4$--$6$\,Hz resting tremor, postural instability \\
		\noalign{\smallskip}\hline\noalign{\smallskip}
		\textbf{2} & \textbf{Exclusion criteria for Parkinson's disease} \\
		  & history of repeated strokes with stepwise progression of parkinsonian features \\
		  & history of repeated head injury \\
		  & history of definite encephalitis \\
		  & oculogyric crises \\
		  & neuroleptic treatment at onset of symptoms \\
		  & sustained remission \\
		  & strictly unilateral features after 3 years \\
		  & supranuclear gaze palsy \\
		  & cerebellar signs \\
		  & early severe autonomic involvement \\
		  & early severe dementia with disturbances of memory, language, and praxis \\
		  & Babinski sign \\
		  & presence of cerebral tumor or communication hydrocephalus on imaging study \\
		  & negative response to large doses of levodopa in absence of malabsorption \\
		\noalign{\smallskip}\hline\noalign{\smallskip}
		\textbf{3} & \textbf{Supportive prospective positive criteria for Parkinson's disease} \\
		  & unilateral onset \\
		  & rest tremor present \\
		  & progressive disorder \\
		  & persistent asymmetry affecting side of onset most \\
		  & excellent response (70--100\,\%) to levodopa \\
		  & severe levodopa-induced chorea \\
		  & levodopa response for 5 years or more \\
		  & clinical course of ten years or more \\
		\noalign{\smallskip}\hline\hline
	\end{tabular}
\end{table*}

\newpage

\section{Overview of acoustic features}

\begin{table*}[ht!]
	\caption{Overview of acoustic features used to quantify phonation in HD.}
	\footnotesize
    \centering
	\label{tab:phonatory_features}
	
	\newcolumntype{s}{>{\raggedright\arraybackslash\hsize=.35\hsize}X}
	\newcolumntype{m}{>{\raggedright\arraybackslash\hsize=.5\hsize}X}
	
	\begin{tabularx}{1.00\textwidth}{mmsX}
		\hline\hline\noalign{\smallskip}
		Specific disorder & Vocal tasks & Acoustic feature & Feature definition \\
		\noalign{\smallskip}\hline\noalign{\smallskip}
		Airflow insufficiency & Expiration with closed or opened lips & MPT & Maximum phonation time, aerodynamic efficiency of the vocal tract measured as the maximum duration of the sustained vowel/consonant. \\
		Irregular pitch fluctuations & Sustained phonation & relF0SD & The standard deviation of fundamental frequency relative to its mean, variation in frequency of vocal fold vibration. \\
		Microperturbations in frequency & Sustained phonation & jitter & Frequency perturbation, the extent of variation of the voice range. Jitter is defined as the variability of the F0 of speech from one cycle to the next. In this case it is implemented as the five-point period perturbation quotient. \\
		Microperturbations in amplitude & Sustained phonation & shimmer & Amplitude perturbation, representing rough speech. Shimmer is defined as the sequence of maximum extent of the signal amplitude within each vocal cycle. In this case implemented as the five-point amplitude perturbation quotient. \\
		Tremor of jaw & Sustained phonation & F1SD, F2SD & The standard deviation of the first (F1) and second (F2) formant. Formants are related to the resonances of the oro-naso-pharyngeal tract and are modified by position of tongue and jaw. \\
		Increased noise & Sustained phonation & mean HNR & Harmonics-to-noise ratio, the amount of noise in the speech signal, mainly due to incomplete vocal fold closure. HNR is defined as the amplitude of noise relative to tonal components in speech. \\
		Aperiodicity & Sustained phonation & DUV & Degree of unvoiced segments, the fraction of pitch frames marked as unvoiced. \\
		\noalign{\smallskip}\hline\hline
	\end{tabularx}
\end{table*}

\newpage

\begin{table*}[htb!]
	\caption{Overview of acoustic features used to quantify articulation in HD.}
	\footnotesize
    \centering
	\label{tab:articulation_features}
	
	\newcolumntype{s}{>{\raggedright\arraybackslash\hsize=.35\hsize}X}
	\newcolumntype{m}{>{\raggedright\arraybackslash\hsize=.5\hsize}X}
	
	\begin{tabularx}{1.00\textwidth}{mmsX}
		\hline\hline\noalign{\smallskip}
		Specific disorder & Vocal tasks & Acoustic feature & Feature definition \\
		\noalign{\smallskip}\hline\noalign{\smallskip}
		Rigidity of tongue and jaw & Monologue, rhythmical units, basic intonation template, reading paragraph, reading with different emotions & F1IR, F2IR, F1SD, F2SD & Interpercentile range (range between 1st and 99th percentile) and standard deviation of the first (F1) and second (F2) formant. Formants are related to the resonances of the oro-naso-pharyngeal tract and are modified by position of tongue and jaw. \\
		Slow alternating motion rate & Diadochokinetic task & DDK rate & Diadochokinetic rate, representing the number of syllable vocalizations per second. \\
		Irregular alternating motion rate & Diadochokinetic task & DDK reg & Diadochokinetic regularity, defined as the standard deviation of distances between following syllables nuclei. \\
		\noalign{\smallskip}\hline\hline
	\end{tabularx}
\end{table*}

\newpage

\begin{table*}[htb!]
	\caption{Overview of acoustic features used to quantify prosody in HD.}
	\footnotesize
    \centering
	\label{tab:prosodic_features}
	
	\newcolumntype{s}{>{\raggedright\arraybackslash\hsize=.35\hsize}X}
	\newcolumntype{m}{>{\raggedright\arraybackslash\hsize=.5\hsize}X}
	
	\begin{tabularx}{1.00\textwidth}{mmsX}
		\hline\hline\noalign{\smallskip}
		Specific disorder & Vocal tasks & Acoustic feature & Feature definition \\
		\noalign{\smallskip}\hline\noalign{\smallskip}
		Monoloudness & Monologue, rhythmical units, basic intonation template, reading paragraph, reading with different emotions & relSEOSD & Speech loudness variation, defined as a~standard deviation of intensity contour relative to its mean. \\
		Monopitch &  & relF0SD & Pitch variation, defined as a~standard deviation of F0 contour relative to its mean. \\
		Inappropriate silences & Reading paragraph & SPIR & Number of speech inter-pauses per minute. \\
		Unnatural speech rate & Basic intonation template, reading paragraph, reading with different emotions & TSR, NSR & T total speech time (TST) is a~duration of the whole speech, and net speech time (NST) is a~duration of speech without pauses. So, the total speech rate (TSR) is defined as a~number of phonemes per TST, and the net speech rate (NSR) as a~number of phonemes per NST. \\
		\noalign{\smallskip}\hline\hline
	\end{tabularx}
\end{table*}

\newpage

\section{Freezing of gait questionnaire}

\begin{table}[htb!]
	\centering
	\begin{threeparttable}
		\caption{Freezing Of Gait Questionnaire template~\cite{Giladi2000}.}
		\label{tab:FOGQ_template}
		\footnotesize
		\centering
		
		\begin{tabularx}{1.00\textwidth}{c X}
			\hline\hline\noalign{\smallskip}
			points & description \\
			\noalign{\smallskip}\hline

				& Q1: \textit{During your \underline{worst} state~--~do you walk:} \\
			0 & Normally \\
			1 & Almost normally~--~somewhat slow \\
			2 & Slow but fully independent \\
			3 & Need assistance or walking aid \\
			4 & Unable to walk \\
			\noalign{\smallskip}\hline

				& Q2: \textit{Are your gait difficulties affecting your daily activities and independence?} \\
			0 & Not at all \\
			1 & Mildly \\
			2 & Moderately \\
			3 & Severely \\
			4 & Unable to walk \\
			\noalign{\smallskip}\hline

				& Q3: \textit{Do you feel that your feet get glued to the floor while walking/turning (freezing)?} \\
			0 & Never \\
			1 & Very rarely~--~about once a~month \\
			2 & Rarely~--~about once a~week \\
			3 & Often~--~about once a~day \\
			4 & Always~--~whenever walking \\
			\noalign{\smallskip}\hline
			
				& Q4: \textit{How long is your \underline{longest} freezing episode?} \\
			0 & Never happened \\
			1 & 1\,--\,2\,s \\
			2 & 3\,--\,10\,s \\
			3 & 11\,--\,30\,s \\
			4 & Unable to walk for more than 30\,s \\
			\noalign{\smallskip}\hline

				& Q5: \textit{How long is your \underline{typical start hesitation} episode (when initiating the first step)?} \\
			0 & None \\
			1 & Takes longer than 1\,s to start walking \\
			2 & Takes longer than 3\,s to start walking \\
			3 & Takes longer than 10\,s to start walking \\
			4 & Takes longer than 30\,s to start walking \\
			\noalign{\smallskip}\hline

				& Q6: \textit{How long is your \underline{typical turning hesitation} episode (freezing when turning)?} \\
			0 & None \\
			1 & Resume turning in 1\,--\,2\,s \\
			2 & Resume turning in 3\,--\,10\,s \\
			3 & Resume turning in 11\,--\,30\,s \\
			4 & Unable to resume turning for more than 30\,s \\
			
			\noalign{\smallskip}\hline\hline
		\end{tabularx}    
	\end{threeparttable}
\end{table}