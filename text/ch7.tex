\chapter[Discussion]{Discussion}
\label{ch7}

This doctoral thesis describes three major studies employed by the author. These studies have been performed over the years $2014$--$2018$ at Department of Telecommunications, Faculty of Electrical Engineering and Communication, Brno University of Technology, in cooperation with national and international partners such as: Applied Neuroscience Research Group, Central European Institute of Technology, Masaryk University, Brno, Czech Republic; First Department of Neurology, St. Anne's University Hospital, Brno, Czech Republic; Neuromorphic Processing Laboratory, Center for Biomedical Technology, Universidad Polit\'{e}cnica de Madrid, Madrid, Spain; and Escola Superior Polit\'{e}cnica, Tecnocampus, Matar\'{o}, Barcelona, Spain. The thesis aimed specifically at investigating possibilities of using quantitative acoustic analysis of dysarthric speech in direction of HD identification and PD assessment (at the baseline as well as in the horizon of two years).

This chapter provides a~discussion about the results of the studies presented in Chapters~\ref{ch4}~\ref{ch5}, and~\ref{ch6}. Next, advantages and disadvantages of the computerized para-clinical approach to diagnosis and assessment of PD based on the quantitative acoustic analysis of dysarthric speech are discussed.

\section{Discussion about the results}
\label{ch7_1}

The first study (Chapter~\ref{ch4}) was focused on complex analysis and accurate identification of dysprosody in HD using three specifically designed speech tasks: emotionally-neutral reading, stress-modified reading, and poem recitation task. This study confirmed the previous findings of reduced variability of intonation \cite{Canter1963, Metter1986, Flint1992, Goberman2005b, Goberman2005d, Rusz2011} and speech intensity \cite{Metter1986, Watson2008, Skodda2011c, Rusz2011, Clark2014} variability, as well as lower speech rate \cite{Weismer1984, Metter1986, Skodda2008, Skodda2011c} in patients with PD. Next, for the first time, it showed a~comparison between neutral, stress-modified, and rhymed speech in terms of HD identification. It is proved that when patients with PD are exposed to additional prosodic demands such as precise control of speech melody during recitation or modification of stress in speech during reading, the underlying prosodic deficits get emphasized, which allows for more accurate identification of HD. The results proposed in this study were also confirmed by permutation test~\cite{Phipson2010} that was used to evaluate the statistical power of the predictions performed by the trained binary classification models. To the best of author's knowledge, this study is the first to use permutation test in the field of acoustic analysis of HD. Furthermore, the results of this study clearly showed that there are some yet to be found gender-related distinctions in the prosodic manifestation of HD that need to be taken into account during the analysis. To sum up, this study is the first to points out to the potential of prosodic analysis of speech signals acquired from stress-modified and rhymed reading to robustly quantify and identify dysprosody~\cite{Galaz2016, Brabenec2017} in HD.

The second study (Chapter~\ref{ch5}) was focused on estimation of PD severity. This study was built on top of the results of the previous one, and extended the prosodic analysis of HD to indirect computerized assessment of motor and non-motor symptoms of PD that are nowadays being commonly evaluated using a~variety of clinical rating scales. This study showed it is possible to use conventional and clinically interpretable acoustic features to estimate values of these rating scales at the baseline (i.\,e. it is possible to assess the severity of PD at the time of the examination). Even though, a~few similar studies aiming at PD assessment have already been employed, this is the first study that uses robust description of dysprosody to assess non-speech symptoms of PD. In addition to that, most of the researchers \cite{Asgari2010, Bayestehtashk2015, Eskidere2012, Peterek2013, Tsanas2010, Tsanas2010a, Tsanas2010b} have been focusing on the estimation of a~single clinical rating scale, namely Unified Parkinson's Disease Rating Scale, part~III: evaluation of motor function~\cite{Fahn1987}. Estimation of other clinical rating scales assessing symptoms such as freezing of gait, sleep disorders, depression, or cognitive deficits has rarely been studied \cite{Mekyska2015, Rektorova2016}. This study also investigated the relationship between HD and various motor and non-motor deficits in PD. It showed that mostly reduced variability of speech intonation and intensity during stress-modified and/or rhymed reading, and speech rate and pausing abnormalities during emotionally-neutral reading are related to other motor symptoms of PD. With respect to non-motor symptoms, speech rate and pausing abnormalities during emotionally-neutral reading was found almost exclusively. To sum up, this study is the first to point out to the potential of prosodic analysis of speech signals acquired from stress-modified, rhymed, and emotionally neutral reading to assess non-speech symptoms of PD. It also shows that in order to robustly assess the severity of PD, reduced variability of intonation and intensity of speech, as well as abnormal speech rate should be taken into account.

The third study (Chapter~\ref{ch6}) was focused on estimation of the changes in freezing of gait occurring with PD in the horizon of two years based on the quantitative acoustic analysis of dysarthric speech in patients with PD. Freezing of gait, as well as other symptoms of PD, is evaluated using a~specialized clinical rating scale~\cite{Giladi2000}, which is composed of several questions (sub-scores) and rated on a~Likert scale. This study was built on top of the results and conclusions summarized in the previously mentioned chapters, and proposed an investigation of the possibilities of using clinically interpretable set of acoustic features quantifying phonation, articulation and prosody using a~variety of speech tasks, according to the recommendations given in~\cite{Brabenec2017}, in direction of complex description of HD. Consequently it showed that indeed it is possible to estimate the change in the severity of gait-related deficits from the baseline description of voice/speech disorders associated with HD, which is an interesting finding that might open the doors for further research and possibly application of this methodology in clinical practice. Furthermore, this thesis also proposed an investigation of pathological mechanisms shared by freezing of gait and HD in PD. So far, only a~few works have addressed this area of research \cite{Giladi2001b, Bartels2003, Moreau2007, Cantiniaux2010, Park2014}. Moreover, none of the works have used such a~complex analytical setup as presented in this study. To specify, this study showed that reduced movement of the tongue and jaw during articulation \cite{Gomez2017, Gomez2017b, Vergara2017}, and speech rate, which in some respect correspond to the intelligibility of speech, are closely related to freezing of gait. To sum up, this study confirms the potential of the acoustic analysis to reveal common pathophysiological mechanisms behind voice/speech disorders in HD and freezing of gait in PD and that it can be used to predict the change in freezing of gait in the horizon of two years.

\section{Advantages and disadvantages}
\label{ch7_2}

The following facts can be identified as being one of the most important and clinically relevant advantages of the para-clinical computerized approach to PD diagnosis and assessment in comparison with the conventional approach that is nowadays being used exclusively in the clinical practice all over the world\footnote{Event though the advantages of the para-clinical approach are presented in terms of the comparison to the clinical one, it is important to note that rather a~fusion of both clinical and para-clinical approaches is considered (see the last paragraphs in Chapter~\ref{ch8}).}:
\begin{enumerate}
	\item \textit{The analysis is free of human subjectivity}. Even if the examination of PD-related symptoms is performed by skilled clinicians, the inherent inter-rater subjectivity plays a~great role in the reliability of the evaluation. The computerized analysis if free of human factor and therefore 100\,\% objective.
	\item \textit{It is possible to quantify deficits not perceptible to humans}. Even the most skilled examiner is subject to limitation of human perception (e.\,g. only sounds in the audible part of its spectrum can be perceived). The computerized acoustic analysis is capable of quantifying a~large variety of characteristics of voice/speech that would otherwise stay neglected.
	\item \textit{It is possible to analyse large amount of data}. Even though today, every database that is used for PD analysis is rather small, it might not be the case in the future. The computerized approach to data analysis provides us with the power to analyse the amount of data that would never be analysable to human beings, especially if time is of the essence.
	\item \textit{Possibility to use modern signal processing and machine learning techniques}. Today, advanced signal processing and machine learning techniques are being applied in almost every field of science. This trend brings new possibilities that have not been available before mainly due to insufficient computational power and unoptimized learning algorithms. 
	\item \textit{It is possible to integrate it into modern wearable devices and generally into the concept of Health 4.0}. Today, there is a~large variety of smart devices such as smart phones, smart watches, etc. that can be used to record and collect various biological signals. Therefore, and integration of the computerized acoustic analysis of dysarthric speech into such devices can be the next step towards improving diagnosis, assessment, and monitoring of PD.
\end{enumerate}
As can be expected, the computerized analysis of PD has several disadvantages as well. In fact, these disadvantages are one of the reasons why this approach has not been applied to clinical practice yet. However, as more studies are employed, more information necessary for this to happen is collected. So, one can say it is only a~matter of time until the analysis of PD is eventually made with the help of signal processing and machine learning. However, until that time, the following facts need to be taken into account:
\begin{enumerate}
	\item \textit{The quality of results depends on the quality of data}. Even the best analytical setup if provided with noisy and disturbed data is likely to produce non-optimal results. Especially the acquisition of voice/speech recordings is sensitive to background noise, quality of microphone, etc.
	\item \textit{The quality of results depends on the quality of speech parametrization}. Until the optimal and robust set of acoustic features, quantifying all important characteristics of voice/speech production deterioration occurring with HD in PD, is found, imperfect predictions will be made.
	\item \textit{Modern machine learning techniques require big data}. Today, there are modern machine learning algorithms such as deep neural networks, etc. that provide state-of-the-art results in many scientific fields. Nevertheless, current databases used for HD-based PD analysis are so far insufficient for such algorithms because of their limited size.
	\item \textit{Acoustic features must be clinically interpretable}. Clinicians are unlikely to trust the results if they are not able to associate the values of the acoustic features with real physiological phenomena inside the human body.
\end{enumerate}