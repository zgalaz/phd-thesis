\chapter[Conclusion]{Conclusion}
\label{ch8}

This doctoral thesis deals with quantitative acoustic analysis of dysarthric speech applied in the field of objective non-invasive computerized diagnosis and assessment of idiopathic PD. The first two parts of the thesis present the theoretical ~background that this thesis was built on. In the first part of the thesis, description of PD along with limitations of the current clinical diagnosis and assessment, and a~proposal of novel para-clinical approach based on the acoustic analysis of dysarthric speech is provided. In the second part, HD is described. This part also summarizes drawbacks of the current state of HD diagnosis and assessment, and provides a~description of the computerized techniques that have been applied by the community of researchers in direction of non-perceptual HD quantification and identification. In the third part, the hypotheses and objectives of this thesis are summarized. 

In the fourth part, a~study focused on robust quantification, description and identification of monopitch, monoloudness and speech rate/pausing abnormalities in patients with PD is presented. In the frame of this study, speech recordings acquired from $98$ PD patients and $51$ healthy speakers were investigated. For this purpose, three specifically-designed speech tasks were recorded to quantify variability of speech melody, speech-stress control and naturalness of speech rate and pausing. With respect to the analyses, a~complex comparison between HC and patients with PD in terms of gender-related distinctions occurring with parkinsonian dysprosody, and a~unique investigation of the possibilities of HD identification using specific prosodic scenarios was performed. In addition, permutation test was applied to evaluate the statistical power of the predictions made by the multivariate classification models trained to discriminate healthy and dysarthric speech.

In the fifth part, a~study focused on computerized and objective assessment of motor and non-motor symptoms of PD based on the quantitative acoustic analysis of dysarthric speech at the baseline is presented. In the frame of this study, speech recordings and clinical data acquired from $72$ PD patients were investigated. For this purpose, the same speech tasks as well as the acoustic features as in the case of the previous study was used. As opposed to the previous study, the correlation analysis aiming at investigating the relationship between dysprosody in HD and other non-speech symptoms of PD was employed. In addition to that, multivariate regression models capable of precise assessment of PD severity were built. These regression models used only the information about prosodic deficits of the patients at the baseline to predict the scores of a~variety of clinical rating scales that are nowadays being commonly used to assess severity of motor and non-motor symptoms of PD.

In the sixth part, a~study focused on computerized and objective assessment of freezing of gait in PD in the horizon of two years based on the quantitative acoustic analysis of dysarthric speech at the baseline is presented. In the frame of this study, a~robust set of acoustic features and speech task quantifying phonation, articulation, prosody, and speech fluency were used. For this purpose, speech recordings and clinical data acquired from $75$ and $41$ PD patients at the baseline and at the follow-up examination were investigated, respectively. In this study, multivariate regression models capable of predicting the change in gait-related deficits in the horizon of two years based on the information about severity of HD at the baseline were built Furthermore, partial correlation analysis was performed in direction of investigating pathological mechanisms shared by HD and freezing of gait in PD.

And finally, in the seventh part of the thesis, a~discussion about the results of the three aforementioned studies that are presented in this thesis is provided. This part also summarizes some of the advantages and disadvantages of the computerized para-clinical approach to diagnosis and assessment of PD based on the quantitative acoustic analysis of voice/speech signals in PD patients suffering from HD.

The main goal of this doctoral thesis was to \textit{investigate possibilities of using quantitative objective evaluation of HD, employing speech parametrization, statistical analyses and machine learning techniques, in direction of PD identification and assessment}. This goal as well as all its objectives were successfully accomplished. More specifically, the following goals were achieved:
\begin{enumerate}
	\item \textit{Robust computerized quantification of HD manifestations in PD} was performed. In the area of phonation, microperturbations in frequency and amplitude, irregular pitch fluctuations, tremor of jaw, increased acoustic noise, insufficient breath support and aperiodicity of voice were quantified. In the area of articulation, rigidity of tongue and jaw, slow alternating motion rate during diadochokinesis and irregular alternating motion rate during diadochokinesis were quantified. In the area of speech prosody, monopitch and monoloudness were quantified. And finally, in the area of speech fluency, inappropriate silences and unnatural speech rate were quantified. These acoustic features provided a~basis for complex computerized description of HD in PD.
	\item \textit{Complex analysis and identification of dysprosody in HD} was employed. To quantify dysprosody in HD, conventional prosodic features, quantifying monopitch, monoloudness and speech rate/pausing abnormalities, were computed from the recordings of three specialized speech tasks: a) poem recitation task (description of flat speech melody), b) stress-modified reading (description of insufficient stress-control), and c) emotionally-neutral reading (description of speech rate/pausing abnormalities). Next, a~comparison between dysarthric and healthy speech was performed. Additionally, multivariate classification models were built to discriminate between PD patients and HC. All of the analyses were employed in the gender-specific setup. Finally, each dimension of dysprosody was evaluated separately as well.
	\item \textit{Assessment of non-speech symptoms of PD at the baseline} was employed. To follow and build on top of the findings and conclusions of the previous study focused on identification of dysprosody in HD, the same acquisition and parameterization setup was used. Here, correlation analysis between prosodic features and values (scores) of a~variety of clinical rating scales assessing motor and non-motor symptoms of PD was performed. Moreover, the computed prosodic features were used to train and evaluate multivariate regression models that were proved to be capable of estimating the scores of these rating scales based solely on the information about the severity of HD at the baseline.
	\item \textit{Assessment of gait freezing in PD in the horizon of two years} was employed. To robustly describe HD in PD, a~large variety of speech tasks such as sustained phonation, expiration, reading, free speech (monologue), diadochokinesis, etc. and acoustic features quantifying all dimensions of speech production were studied. These features were consequently used to to train and evaluate multivariate regression models that were proved to be capable of predicting the change in the freezing of gait occurring with PD in the horizon of two years based solely on the information about the severity of HD at the baseline.
	\item \textit{Analysis of pathological mechanism shared by HD and gait freezing in PD} was employed. To investigate if there are pathological mechanisms shared by HD and freezing of gait in PD, partial correlation analysis controlling for the effect of other confounding factors such as age, gender, dopaminergic medication, etc., between the acoustic features and values of the specialized clinical rating scale assessing gait-related deficits in PD was performed. This analysis pointed out to some interesting facts about the relationship between HD and gait freezing in patients with PD.
\end{enumerate}

Regarding the future direction of the research described in this thesis, application of the presented methodology for assessing of other common parkinsonian symptoms such as depression or cognitive deficits at the baseline as well as in the direction of two years is considered. Moreover, investigation of pathological mechanisms shared by HD and other symptoms of PD besides freezing of gait is considered. Next, application of quantitative acoustic analysis of dysarthric speech in direction of tuning the parameters of novel perspective PD treatment methods such as rTMS is considered as well. And finally, the ultimate goal behind this research is the fusion of clinical and paraclinical methodology in order to develop and evaluate a~decision support system that would help clinicians with diagnosis, assessment, and monitoring of PD.