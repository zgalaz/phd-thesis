\chapter*{Introduction}
\phantomsection
\addcontentsline{toc}{chapter}{Introduction}

Nowadays, we observe two main phenomena in the genesis of Parkinson's disease (PD). Namely the progressive degeneration of dopaminergic neurons in the \textit{substancia nigra pars compacta} of the cerebra, and/or development of $\alpha$-synuclein-containing Lewy bodies within the surviving neurons. The associated motor symptoms such as tremor at rest, progressive bradykinesia, muscular rigidity, postural instability, gait freezing, voice/speech disorders, etc., and non-motor symptoms such as behavioural alternations, reduction of cognitive abilities, sleep disturbances, anxiety, depression, etc. have a~detrimental impact on patients' health, physical and mental condition, social life, independence, and quality of life in general. Typically, PD is rare in young population and its prevalence rate grows with the advancement of a~person's age. That's why it is mostly diagnosed in persons aged over $60$ years. But, before the conclusive clinical diagnosis is finally made, there is a~long period of the development of the underlying neurodegenerative process behind the disease, slowly but surely worsening the severity of its symptoms. Thus, one might be asking, what is necessary for the diagnosis of PD to be made?

At some point, the cardinal motor symptoms are the ones that first bring patients to a~hospital searching for help, and even though the disease gets finally diagnosed, at this stage, most of the dopaminergic neurons have already been damaged, unfortunately. As one can imagine, the conventional clinical diagnosis of PD is therefore based on the presence of the above-mentioned cardinal motor symptom. Nevertheless, the presence of these symptoms is still not enough, and other criteria such as the short-term positive response to dopaminergic (anti-parkinsonian) medication, and many others have to be met. It is therefore obvious that the diagnosis of PD is not an easy task. In fact, even today, an objective diagnostic test which allows a~definitive 100\,\% accurate diagnosis of this disease has not been developed. Thus, clinicians are forced to use a~battery of tests, heuristics, biomarkers, and inclusion/exclusion criteria to make the diagnosis as accurate as possible. Another drawback of the current state of affairs is that this set of examinations has to be taken in the medical environment under the supervision of skilled clinician/s, which is logistically demanding, costly and time-consuming. Not to mention the fact that the disease does not have to be diagnosed at the first trial. It is often the case that prior the diagnosis elderly people have to visit the hospital several times, which makes this whole process even more problematic.

Today, we are living in the era of modern technologies, smart devices, internet of things, etc. Even though older population might not be adopted to such a~technological advancement, younger people essentially grow up surrounded by it. Nowadays, smart phones, smart watches and other devices can be easily used to record a~large variety of biological signals such voice/speech, movements of hands, gait, heart rate, and many more. With the previously mentioned facts in mind, it seems that one of the major obstacles of PD diagnosis is the lack of data available for the clinicians. Therefore, these modern devices could be potentially used to collect a~large amount of data without necessity of the patient's presence at the clinic or any specialized supervision. Such data could be securely transmitted and stored on cloud, where only authorized persons could be allowed to access them. With this approach, clinicians would be provided with an additional information about the medical condition of their patients that could definitely help with their decision making that is related to diagnosis, assessment, treatment and/or monitoring of the disease. Imagine a~system that would be able to access and process all clinical data (data acquired by a~doctor as well as those acquired by a~variety of specialized devices such as those discussed above) available for a~patient. The large scale of data that would be available could provide such a~system with the power to use advanced signal processing techniques to quantify and describe properties of the acquired biological signals that might even lay beyond human perception. Next, modern machine learning algorithms, statistical analyses and visualization methods could be applied to provide clinicians with powerful reports about the current state of biomarkers and their evolution in time, and so on and so forth. It is evident that not only doctors, but also patients themselves would benefit from such information. However, to reach that point, relationship between properties of these biological signals and other clinical symptoms of PD needs to be investigated and fully understood.

Speech is the most natural way of communication. In most cases, people use it without problems. However, when a~disorder such as PD comes into play, speech disorder named hypokinetic dysarthria (HD) gets involved. The associated voice/speech deviations in the early stages of the disease are very hard to be clearly perceived. In addition to that, patients themselves are in most cases not aware of their handicap, and the perception of the changes in their voice and speech is different than the one reported by their family and relatives. But in general, and depending on the stage of the disease, at some point, speech communication difficulties will eventually come. In fact, HD is one of the most disabling symptoms of PD that occurs in most of the patients suffering from it, and therefore, even though HD has a~detrimental impact on the patient's quality of life, it might be used as a~rich source of information for its diagnosis, assessment and monitoring.

Taking into account all the previously mentioned information, it can be hypothesized that quantitative acoustic analysis of voice/speech signals might be used to quantify different vocal manifestations of HD. Therefore, the main aim of this doctoral thesis is to \textit{investigate possibilities of using a~combination of speech parametrization and machine learning techniques for remote, computerized, para-clinical and objective PD diagnosis and assessment}. The goals of this thesis are: (i) to use modern speech parameterization techniques to quantify HD manifestations in the area of phonation, articulation, prosody and speech fluency, (ii) to use quantitative acoustic analysis of dysarthric speech to identify HD, (iii) to use acoustic analysis of dysarthric speech to objectively and indirectly assess severity of PD at the baseline, (iv) to use acoustic analysis of dysarthric speech to predict the change in the severity of freezing of gait in PD in the horizon of two years, and (v) investigate pathological mechanisms shared by HD and freezing of gait in PD.

The thesis is structured as follows. Chapter~\ref{ch1} introduces the state of knowledge in the field of PD analysis and points out to limitations of its current clinical diagnosis and assessment. Next, it provides a~brief proposal of the non-invasive para-clinical computerized approach taking advantage of modern digital signal processing algorithms and machine learning techniques. Chapter~\ref{ch2} introduces the state of knowledge in the field of HD analysis and points out to limitations of its current clinical diagnosis and assessment. Next, it provides a~description of the quantitative acoustic analysis of voice/speech signals for describing voice/speech-related abnormalities that may not be audible to humans. It also summarizes the speech parametrization setup that have been commonly used in this field of science. Chapter~\ref{ch3} provides a~description of the hypotheses and goals of this thesis. Chapter~\ref{ch4} summarizes the results of a~study focused on robust quantification and identification of dysprosody in HD using conventional clinically interpretable acoustic features and three speech tasks specifically designed to describe prosodic manifestations of HD. It also presents the results of a~statistical analysis devoted to differentiation between healthy and dysarthric speech in terms of speech melody and speech rate abnormalities. Furthermore, presence of gender-specific patterns of dysprosody in HD is investigated as well. Chapter~\ref{ch5} summarizes the results of a~study focused on objective computerized assessment of PD severity based on the acoustic analysis of dysarthric speech. It also presents the results of a~correlation analysis between acoustic features quantifying dysprosody in HD and a~variety of clinical rating scales that are nowadays being commonly used to assess motor and non-motor symptoms of PD. Next, it demonstrates it is possible to use acoustic analysis of voice/speech signals to estimate the values of these rating scales at the baseline. Chapter~\ref{ch6} summarizes the results of a~study focused on the estimation of the changes in freezing of gait (FOG) occurring with PD in the horizon of two years based on the quantitative acoustic analysis of dysarthric speech. It also presents the results of a~partial correlation analysis aiming at investigating pathological mechanisms shared by HD and FOG in PD. Next, it demonstrates it is possible to use acoustic analysis of voice/speech signals to predict the progress of FOG in the horizon of two years. And finally, Chapter~\ref{ch7} provides discussion, and Chapter~\ref{ch8} summarizes the thesis.